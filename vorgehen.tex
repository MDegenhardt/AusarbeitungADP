\chapter{Vorgehen}
In diesem Kapitel soll das Vorgehen zur Erstellung dieser Arbeit und zur Erlangung der vorgestellten Ergebnisse aufgezeigt werden. \\
Der Startpunkt der Arbeit ist ein Geometriemodell der 1 1/2 stufigen Aachenturbine. Darauf aufbauend werden erste Test-Simulationen durchgeführt, um sich mit der Thematik und der verwendeten Software vertraut zu machen. Die nächste Aufgabe besteht darin ein geeignetes Gitter für die endgültigen Berechnungen und Auswertungen zu finden. Hierbei wird zunächst der strukturierte Vernetzer AutoGrid von Numeca verwendet.\\
Anschließend wird das Augenmerk auf die Verbindungsstellen der einzelnen Domänen, Interfaces, gelegt. Hierzu dient ein einfaches Kanalmodell, bestehend aus einem stationären Kanal gefolgt von einem rotierenden Kanal, mit ähnlichen Abmessungen der Aachenturbine. Die Geometrieerstellung erfolgt mit Siemens NX, die Netzgenerierung mittels ANSYS ICEM CFD.\\ 
Die Auswerung der einzelnen Größen erfolgt mit MATLAB, hierfür dient eine während dieser Arbeit entwickelte Applikation zum einfacheren Einlesen der Ergebnisdaten und deren Visualisierung.    

\section*{Aufbau der Arbeit}
Beginnend mit Kapitel \ref{cha:grundlagen} werden Grundlagen zu Turbomaschinen und grundsätzliche Definitionen zur Bestimmung des Wirkungsgrades eingeführt.\\
In Kapitel \ref{cha:aachen} wird die Erstellung der Gitter, des Setups, sowie ein Vergleich der Wirkungsgrade verschiedener Konfigurationen für strukturierte und unstrukturierte Netze der verwendeten Aachen-Turbine vorgestellt.\\
Der Einfluss des Interfaces an den Übergängen, von zum Beispiel Stator zu Rotor, wird in Kapitel \ref{cha:kanal} untersucht. Hierbei werden verschiedene Netzkonfigurationen, sowie Randbedingungen untersucht, um den Einfluss des Interfaces auf den Wirkungsgrad abzuschätzen.\\
Im letzten Kapitel wird das in MATLAB implementierte Auswertungstool der Gitterstudie, dessen Funktionsumfang und Bedienung vorgestellt. 