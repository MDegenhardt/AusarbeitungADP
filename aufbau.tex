\section{Aufbau der Arbeit}
Beginnend mit dem Kapitel \ref{cha:wgberechnung} werden einige grundsätzliche Definitionen zur Bestimmung des Wirkungsgrades für Turbomaschinen eingeführt.\\
In Kapitel \ref{cha:aachen} wird die Erstellung der Gitter, des Setups, sowie ein Vergleich der Wirkungsgrade verschiedener Konfigurationen, für die verwendete Aachenturbine vorgestellt.\\
Der Einfluss des Interfaces an Domänenverbindungen wird in Kapitel \ref{cha:kanal} untersucht. Hierbei werden verschiedene Netzkonfigurationen, sowie Randbedingungen untersucht um eine Möglichkeit zu untersuchen den Einfluss des Interfaces auf den Wirkungsgrad abzuschätzen.\\
Im \ref{cha:auswertungstool}. und letzten Kapitel wird schließlich das in MATLAB Implementierte Auswertungstool der Gitterstudie, dessen Funktionsumfang und Bedienung, vorgestellt. 