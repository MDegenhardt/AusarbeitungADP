\section*{Aufbau der Arbeit}
Beginnend mit Kapitel \ref{cha:grundlagen} werden Grundlagen zu Turbomaschinen und grundsätzliche Definitionen zur Bestimmung des Wirkungsgrades eingeführt.\\
In Kapitel \ref{cha:aachen} wird die Erstellung der Gitter, des Setups, sowie ein Vergleich der Wirkungsgrade verschiedener Konfigurationen für strukturierte und unstrukturierte Netze der verwendeten Aachen-Turbine vorgestellt.\\
Der Einfluss des Interfaces an den Übergängen, von zum Beispiel Stator zu Rotor, wird in Kapitel \ref{cha:kanal} untersucht. Hierbei werden verschiedene Netzkonfigurationen, sowie Randbedingungen untersucht, um den Einfluss des Interfaces auf den Wirkungsgrad abzuschätzen.\\
Im letzten Kapitel wird das in MATLAB implementierte Auswertungstool der Gitterstudie, dessen Funktionsumfang und Bedienung vorgestellt. 