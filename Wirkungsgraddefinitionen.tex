\chapter{Wirkungsgradberechnung}
\label{cha:wgberechnung}
In diesem Kapitel werden die verschiedenen Möglichkeiten zur Berechnung des Wirkungsgrades bei Turbinen vorgestellt. Diese werden im Rahmen der Simulation der Aachen-Turbine getestet und die resultierenden Wirkungsgrade miteinander verglichen.
%__________________________
\section{Wirkungsgrad allgemein}
Im Allgemeinen lautet die Gleichung für den Wirkungsgrad $\eta$ bei Turbinen 
\begin{equation}
\label{eq:wgallgemein}
\eta =\frac{P}{\Delta H_{t_{is}}}
\end{equation}
mit der erzeugten der isentropen Totalenthalpiedifferenz $\Delta H_{t_{is}}$ und der erzeugten Leistung $P$.
Die isentrope Enthalpiedifferenz $\Delta H_{t_{is}}$ wird nach
\begin{equation}
\label{eq:wgnenner}
\Delta H_{t_{is}} = \dot m \cdot c_p \cdot T_{t_{inlet}} \cdot \left[ \left( \frac{p_{t_{outlet}}}{p_{t_{inlet}}}\right)^\frac{\gamma-1}{\gamma}-1\right]
\end{equation}
mit dem Massenstrom $\dot m$, der spezifischen Wärmekapazität bei konstantem Druck $c_p$ (siehe Berechnungsweise Abschnitt \ref{subsec:spezWK}), der Totaltemperatur am Inlet $T_{t_{inlet}}$, dem Totaldruck am Inlet $p_{t_{inlet}}$, dem Totaldruck am Outlet $p_{t_{outlet}}$ und dem Isentropenexponenten $\gamma$ berechnet.\newline Für die Berechnung der erzeugten Leistung $P$ gibt es verschiedene Möglichkeiten, die im folgenden Abschnitt vorgestellt werden.
%__________________________
\section{Wirkungsgradberechnungsformeln}
\label{sec:wirkungsgrade}
Die erzeugte Leistung $P$ im Zähler der Gleichung \ref{eq:wgallgemein} für den Wirkungsgrad  lässt sich unter anderem mit einer der drei folgenden Gleichungen berechnen:\newline
Mit Hilfe der tatsächlichen Totaltemperaturdifferenz $\Delta T_t$ nach
\begin{equation}
\label{eq:wgzaehlertt}
P_{\Delta T_t} = \dot m \cdot c_p \cdot \Delta T_t = \dot m \cdot c_p \cdot \left( T_{t_{inlet}}-T_{t_{outlet}} \right),
\end{equation}
mithilfe der Totalenthalpie am Inlet $h_{t_{inlet}}$ und Outlet $h_{t_{outlet}}$ (direkt aus CFX entnommen) nach
\begin{equation}
\label{eq:wgzaehlerht}
P_{\Delta h_t} = \dot m \cdot \Delta h_t = \dot m \cdot \left( h_{t_{inlet}}-h_{t_{outlet}} \right)
\end{equation}
oder mithilfe des Momentes $M_{Rotor}$ um die Rotationsachse an Blade und Hub im Rotor, der Drehzahl des Rotors $n_{Rotor}$ und der Winkelgeschwindigkeit $\omega$ nach 
\begin{equation}
\label{eq:wgzaehlertorque}
P_{torque} = M_{Rotor} \cdot n_{Rotor} \cdot \omega
\end{equation}
Im kommenden Abschnitt wird die Berechnungsmöglichkeiten von $c_p$ aus den Gleichungen \ref{eq:wgnenner} und\ref{eq:wgzaehlertt} näher erläutert, da diese Auswirkungen auf die Wirkungsgradberechnung haben.
\subsection{Spezifische Wärmekapazität}
\label{subsec:spezWK}
Die spezifische Wärmekapazität bei konstantem Druck $c_p$ ist eine temperaturabhängige Größe. Wenn die Temperaturdifferenz zwischen In- und Outlet sehr groß ist, verändert sich  zwischen In- und Outlet wesentlich und kann nicht mehr als konstant angenommen werden. Die temperaturabhängige Wärmekapazität $c_p$ lässt sich mit dem folgenden Polynom aus der Temperatur $T$ berechnen.  
\begin{equation}
\label{eq:cppolynom}
c_p = \frac{a\cdot T^4-b\cdot T^3+c\cdot T^2-d\cdot T+e}{f}\frac{J} {kg K}
\end{equation}
Die Konstanten $a$ bis $f$ sind der folgenden Tabelle \ref{tab:cpparameter} zu entnehmen.
\begin{table}[H]
\centering
\caption{Konstanten für die Berechnung von $c_p$}
\begin{tabular}{ c| c|c|c|c|c}
$a$&$b$&$c$&$d$&$e$&$f$\\
\hline
$0.12934K^{-4}$&$596.633K^{-3}$&$933833K^{-2}$&$373,61\cdot10^6K^{-1}$&$105,01\cdot10^{10}$&$10^9$\\
\end{tabular}
\label{tab:cpparameter}
\end{table}
Bei der Berechnung der isentropen Enthalpiedifferenz $\Delta H_{t_{is}}$ aus Gleichung \ref{eq:wgnenner} wurde $c_p$ nach Gleichung \ref{eq:cppolynom} separat am In-/Outlet, aus dem arithmetischen Mittel der beiden Größen und in Abhängigkeit der isentropen Temperatur im Outlet berechnet, um den Einfluss der Berechnungsweise von $c_p$ auf die Berechnung des Wirkungsgrades zu analysieren.\newline
Die verschiedenen Wirkungsgraddefinitionen in Abschnitt \ref{sec:wirkungsgrade} und Berechnungsweisen von $c_p$ wurden in CFX implementiert und miteinander verglichen. Das Ergebnis dieses Vergleichs wird im nächsten Abschnitt dargestellt.
%__________________________
\section{Wirkungsgrade bei der Aachen-Turbine}
\label{sec:wgaachen}
Bei der Aachen-Turbine ergaben sich je nach Berechnungsart folgende Werte für den Wirkungsgrad:
\begin{table}[H]
\centering
\caption{Wirkungsgrad bei der Aachen-Turbine}
\begin{tabular}{ c| c}
Berechnungsformel & $\eta$ \\
\hline
$\eta_{\Delta T_t}$ mit $c_p$ konstant& 86\% \\
$\eta_{\Delta T_t}$ mit $c_p(T)$& 86\% \\
$\eta_{\Delta h_t}$& 87\% \\
$\eta_{torque}$& 85\% \\
\end{tabular}
\label{tab:wgaachen}
\end{table}
Es ist zu sehen, dass .....





%%% Local Variables: 
%%% mode: latex
%%% TeX-master: "main"
%%% End: 


