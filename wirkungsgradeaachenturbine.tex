\section{Vergleich der Wirkunsgrade}
Bei der Aachen-Turbine ergaben sich je nach Berechnungsart folgende Werte für den Wirkungsgrad:
\begin{table}[H]
	\centering
	\caption{Wirkungsgrad bei der Aachen-Turbine}
	\begin{tabular}{ c| c | c}
Berechnungsformel	&	$\eta_{strukturiert}$	&	$\eta_{unstrukturiert}$	\\
\hline
1 Stufe	&		&		\\
\hline
$\eta_{CFX}$	&	$92,33\%$	&	$92,23\%$	\\
$\eta_{c_p, T_t}$	&	$92,44\%$	&	$149,3\%$	\\
$\eta_{torque}$	&	$92,87\%$	&	$90,58\%$	\\
\hline
1,5 Stufen 	&		&		\\
\hline
$\eta_{CFX}$	&	$86,42\%$	&	$87,05\%$	\\
$\eta_{c_p, T_t}$	&	$86,48\%$	&	$87\%$	\\
$\eta_{torque}$	&	$87.05\%$	&	$85,58\%$	\\

	\end{tabular}
	\label{tab:wgaachen}
\end{table}
In Tabelle \ref{tab:wgaachen} werden die Wirkunsgrade des strukturierten und des unstrukturierten Gitters dargestellt. Diese weichen höchstens um $1,7\%$ voneinander ab. Es ist zu erkennen, dass die Wirkunsgrade im strukturierten Fall bei einer Stufe bis zu zwei Prozentpunkten größer sind. Bei der Betrachtung von 1,5 Stufen ist dagegen zu erkennen, dass die Wirkungsgrade im strukturierten Fall um $0,6$ Prozentpunkte kleiner sind.\\
Dies kann an dem Übergang vom Rotor in den zweiten Stator liegen. Dort findet ein Wechsel des Gitters von unstrukturiert nach strukturiert statt.\\
Allerdings ist die durchgeführte Gitterstudie im unstrukturierten Fall nicht zufriedenstellend durchgeführt, was in den Abbildungen \ref{fig:gitterunstrukturiert1stufe} und\ref{fig:gitterunstrukturiert15stufen} zu erkennen ist. Daher fällt ein Vergleich der beiden Gittertypen schwer und ist nicht aussagekräftig.
