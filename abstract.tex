%******  Abstract **************

\begin{abstract}
\setlength{\baselineskip}{11pt}
Die Zuverlässigkeit numerischer Vorhersagen der Wirkungsgrades von Turbinenstufen ist ein wichtiger Bestandteil diverser Forschungsaktivitäten wie Parameterstudien und Optimierungen. Mit zunehmender Komplexität der numerischen Modelle und der simulierten physikalischen Phänomene nimmt jedoch ach die Fehleranfälligkeit der numerischen Vorhersagen zu. Für eine sinnvolle Interpretation der numerischen Ergebnisse ist die Kenntnis der Einflussgrößen daher sehr wichtig. Um die eben genannte numerische Fehleranfälligkeit genauer zu untersuchen werden zwei Testfälle herangezogen. Zum Einen wird eine 1 1/2 Stufige Aachenturbine simuliert, des weiteren werden die Verbindungsstellen einzelner Domänen, sogenannte Interfaces, in einem einfachen Modell eines zweigeteilten Kanals untersucht. 

\begin{comment}
Die Ergebnisse von RANS sind heute immer weniger zufriedenstellend, da die verbesserte Rechenleistung für komplexere Verfahren ausreicht. LES dagegen ist für viele Fälle in der Industrie noch zu aufwändig und rechenintensiv. Eine hybride Variante, welche die Vorteile beider Verfahren kombiniert, soll diese Lücke dahingehend schließen, dass der benötigte Rechenaufwand den gegebenen Kapazitäten angeglichen wird.\\ \\
In diesem Sinne wird in dieser Arbeit eine DDES eines stationären und eines oszillierenden Zylinders bei einer Reynoldszahl von $Re = 10000$ mit dem Inhouse Strömungslöser FASTEST des Fachgebiets Numerische Berechnungsverfahren der TU Darmstadt durchgeführt. Als Turbulenzmodell wird das $\zeta$-$f$-Turbulenzmodell nach Hanjali\'{c} (2004)  verwendet. Ein besonderes Augenmerk wird auf die Entwicklung des Auftriebs- und Widerstandsbeiwertes in Abhängigkeit der Oszillationsfrequenz des Zylinders gelegt. Für das Auswerten der Ergebnisse stehen Experimente, insbesondere von Gopalkrishnan (1993) und eine DNS von Dong und Karniadakis (2005) , an der sich der Simulationsaufbau orientiert, zu Verfügung. Die in dieser Arbeit berechneten Werte der DDES stimmen gut mit diesen vorliegenden Ergebnissen aus Experiment und DNS überein.
\setlength{\baselineskip}{13.2pt}
\end{comment}

\end{abstract}