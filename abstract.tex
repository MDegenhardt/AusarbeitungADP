%******  Abstract **************

\begin{abstract}
%\setlength{\baselineskip}{11pt}
Die Zuverlässigkeit numerischer Vorhersagen des Wirkungsgrades von Turbinenstufen ist ein wichtiger Bestandteil diverser Forschungsaktivitäten wie Parameterstudien und Optimierungen. Mit zunehmender Komplexität der numerischen Modelle und der simulierten physikalischen Phänomene nimmt jedoch auch die Fehleranfälligkeit der numerischen Vorhersagen zu. Für eine sinnvolle Interpretation der numerischen Ergebnisse ist die Kenntnis der Einflussgrößen daher sehr wichtig. Um die eben genannte numerische Fehleranfälligkeit genauer zu untersuchen, werden zwei Testfälle herangezogen. Zum einen wird eine 1 1/2 stufige Aachenturbine simuliert, zum anderen werden die Verbindungsstellen einzelner Domänen, sogenannte Interfaces, in einem einfachen Modell eines zweigeteilten Kanals untersucht. Zunächst erfolgt eine Gitterstudie der zu untersuchenden Aachenturbine, gefolgt von der Untersuchung von Einflussfaktoren auf den Wirkungsgrad.

\end{abstract}